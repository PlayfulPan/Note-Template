% ======================================================== %
% -------------------- Theorem Setup --------------------- %
% ======================================================== %

%------------------
% Setup
%------------------

\makeatletter

\definecolor{DefinitionColor}{HTML}{006400}
\NewTcbTheorem[number within=section, crefname={definition}{definitions}]{Definition}{Definition}{
	enhanced,
	breakable,
    before skip=4mm,
    after skip=1cm,
    colback=DefinitionColor!5,
    colframe=DefinitionColor,
    boxrule=0.5mm,
    attach boxed title to top left={xshift=1cm,yshift*=1mm-\tcboxedtitleheight},
    varwidth boxed title*=-3cm,
    separator sign none,
    description delimiters parenthesis,
    boxed title style={
        frame code={
            \path[fill=tcbcolback!30!black]
            ([yshift=-1mm,xshift=-1mm]frame.north west)
            arc[start angle=0,end angle=180,radius=1mm]
            ([yshift=-1mm,xshift=1mm]frame.north east)
            arc[start angle=180,end angle=0,radius=1mm];
            
            \path[left color=DefinitionColor!60!black,right color=DefinitionColor!60!black,middle color=DefinitionColor!80!black]
            ([xshift=-2mm]frame.north west) --
            ([xshift=2mm]frame.north east)[rounded corners=1mm] --
            ([xshift=1mm,yshift=-1mm]frame.north east) --
            (frame.south east) --
            (frame.south west) --
            ([xshift=-1mm,yshift=-1mm]frame.north west)
            [sharp corners]--
            cycle;
        },
        interior engine=empty,
    },
    fonttitle = \bfseries\sffamily,
	description font = \mdseries,
    title={#2},
    #1
}{def}


\definecolor{TheoremColor}{HTML}{00007B}
\newtcbtheorem[use counter from=Definition, crefname={theorem}{theorems}]{Theorem}{Theorem}
{%
	enhanced,
	breakable,
	colback=TheoremColor!5,
	frame hidden,
	boxrule = 0sp,
	borderline west = {2pt}{0pt}{TheoremColor},
	sharp corners,
	detach title,
	before upper = \tcbtitle\par\smallskip\itshape,
	coltitle = TheoremColor,
	fonttitle = \bfseries\sffamily,
	description font = \mdseries,
	separator sign none,
    description delimiters parenthesis,
	segmentation style={solid, TheoremColor},
}
{th}

\definecolor{PropositionColor}{HTML}{4682B4}
\newtcbtheorem[use counter from=Definition, crefname={proposition}{propositions}]{Proposition}{Proposition}
{%
	enhanced,
	breakable,
	colback = PropositionColor!5,
	frame hidden,
	boxrule = 0sp,
	borderline west = {2pt}{0pt}{PropositionColor},
	sharp corners,
	detach title,
	before upper = \tcbtitle\par\smallskip\itshape,
	coltitle = PropositionColor,
	fonttitle = \bfseries\sffamily,
	description font = \mdseries,
	separator sign none,
    description delimiters parenthesis,
	segmentation style={solid, PropositionColor},
}
{prop}

\providecommand{\qedsymbol}{}
\renewcommand{\qedsymbol}{\raisebox{0.08em}{\ensuremath{\mdwhtsquare}}}

\renewtcolorbox{proof}[1][]{
    enhanced,
	breakable,
    detach title,
    frame hidden,
    before upper = \tcbtitle~,
    fonttitle = \bfseries\rmfamily,
    coltitle = black,
    title = Proof.,
    after upper = \qed,
    colback = white,
    size = minimal,
    #1,
}

\definecolor{SummaryColor}{HTML}{4B0082}
\newtcolorbox{reference}[2][]{
    enhanced,
	breakable,
    before skip=2mm,
    after skip=2mm,
    colback=SummaryColor!5,
    colframe=SummaryColor,
    boxrule=0.5mm,
    attach boxed title to top left={xshift=1cm,yshift*=1mm-\tcboxedtitleheight},
    varwidth boxed title*=-3cm,
    boxed title style={
        frame code={
            \path[fill=tcbcolback!30!black]
            ([yshift=-1mm,xshift=-1mm]frame.north west)
            arc[start angle=0,end angle=180,radius=1mm]
            ([yshift=-1mm,xshift=1mm]frame.north east)
            arc[start angle=180,end angle=0,radius=1mm];
            
            \path[left color=SummaryColor!60!black,right color=SummaryColor!60!black,middle color=SummaryColor!80!black]
            ([xshift=-2mm]frame.north west) --
            ([xshift=2mm]frame.north east)[rounded corners=1mm] --
            ([xshift=1mm,yshift=-1mm]frame.north east) --
            (frame.south east) --
            (frame.south west) --
            ([xshift=-1mm,yshift=-1mm]frame.north west)
            [sharp corners]--
            cycle;
        },
        interior engine=empty,
    },
    fonttitle=\bfseries,
    title={#2},
	subtitle style={
		enhanced,
		colback=SummaryColor!5,
		colupper=black,
	},
    #1
}

\newtcolorbox{summary}[1][]{
    enhanced,
    title=My title,
    colframe=blue!50!black,
    colback=blue!10!white,
    colbacktitle=blue!5!yellow!10!white,
    fonttitle=\bfseries,
    coltitle=black,
    attach boxed title to top center={
        yshift=-0.25mm-\tcboxedtitleheight/2,
        yshifttext=2mm-\tcboxedtitleheight/2
        },
    boxed title style={
        boxrule=0.5mm,
        frame code={
            \path[tcb fill frame] ([xshift=-4mm]frame.west) --
            (frame.north west) -- 
            (frame.north east) -- 
            ([xshift=4mm]frame.east) --
            (frame.south east) --
            (frame.south west) -- 
            cycle;
            },
        interior code={
            \path[tcb fill interior] ([xshift=-2mm]interior.west) --
            (interior.north west) --
            (interior.north east) --
            ([xshift=2mm]interior.east) --
            (interior.south east) --
            (interior.south west) -- cycle;
            }
        },
    #1,
}




\usetikzlibrary{arrows,calc,shadows.blur}
\newtcolorbox{note}[1][]{
	enhanced jigsaw,
	colback=gray!20!white,
	colframe=gray!80!black,
	boxrule=1pt,
	title=Note,
    fonttitle=\bfseries,
	halign title=flush center,
	coltitle=black,
	breakable,
	drop shadow=black!50!white,
	attach boxed title to top left={
        xshift=1cm,
        yshift=-\tcboxedtitleheight/2,
        yshifttext=-\tcboxedtitleheight/2
    },
	minipage boxed title=1.5cm,
	boxed title style={
			colback=white,
			size=fbox,
			boxrule=1pt,
			boxsep=2pt,
			underlay={
					\coordinate (dotA) at ($(interior.west) + (-0.5pt,0)$);
					\coordinate (dotB) at ($(interior.east) + (0.5pt,0)$);
					\begin{scope}
						\clip (interior.north west) rectangle ([xshift=3ex]interior.east);
						\filldraw [white, blur shadow={shadow opacity=60, shadow yshift=-.75ex}, rounded corners=2pt] (interior.north west) rectangle (interior.south east);
					\end{scope}
					\begin{scope}[gray!80!black]
						\fill (dotA) circle (2pt);
						\fill (dotB) circle (2pt);
					\end{scope}
				},
		},
	#1,
}


\definecolor{NotationColor}{HTML}{6F4E37}


\newtcolorbox{notation}[1][]{
    enhanced,
	breakable,
	colback=gray!20!white,
	colframe=NotationColor,
	attach boxed title to top left={yshift*=-\tcboxedtitleheight},
	fonttitle=\bfseries,
	title={Context and Notation},
	boxed title size=title,
	boxed title style={
        sharp corners,
        rounded corners=northwest,
        colback=tcbcolframe,
        boxrule=0pt,
	},
	underlay boxed title={
		\path[fill=tcbcolframe]
        (title.south west) --
        (title.south east)
		to[out=0, in=180]
        ([xshift=5mm]title.east) --
        (title.center-|frame.east)
        [rounded corners=\kvtcb@arc] |-
        (frame.north) -|
        cycle;
	},
    #1,
}
\makeatother

