% ======================================================== %
% -------------------- Theorem Setup --------------------- %
% ======================================================== %


%These patch the commands for the theorem environments to allow for margin changes (Cleveref redefines the commands, if there wasn't cleveref, the command to patch would be \@thm)

\makeatletter

\patchcmd{\cref@thmoptarg}{\trivlist}{\list{}{\leftmargin=\thm@leftmargin\rightmargin=\thm@rightmargin}}{}{} 
\patchcmd{\cref@thmnoarg}{\trivlist}{\list{}{\leftmargin=\thm@leftmargin\rightmargin=\thm@rightmargin}}{}{}
\patchcmd{\@endtheorem}{\endtrivlist}{\endlist}{}{}

\newlength{\thm@leftmargin}% Default parameters for theorem environments
\setlength{\thm@leftmargin}{\parindent}

\newlength{\thm@rightmargin}
\setlength{\thm@rightmargin}{0pt}

\newlength{\thm@parindent}
\setlength{\thm@parindent}{\parindent}

\newlength{\thm@parskip}
\setlength{\thm@parskip}{\parskip}

\newlength{\thm@postheadspace}
\setlength{\thm@postheadspace}{1ex}

\newlength{\thm@spacing}
\setlength{\thm@spacing}{2\topsep}

\newcommand{\xdeclaretheorem}[4][]{% Declare a theorem with custom margins
    \newenvironment{#4}
    {\thm@leftmargin=#2\relax\thm@rightmargin=#3\relax\begin{#4INNER}}
    {\end{#4INNER}}%
    \declaretheorem[#1]{#4INNER}%
}

% ======================================================== %
% -------------------- Theorem Styles -------------------- %
% ======================================================== %

\declaretheoremstyle[% Italic style
    spaceabove=\thm@spacing,
    spacebelow=\thm@spacing,
    notefont=\bfseries,
    postheadspace=\thm@postheadspace,
    bodyfont=\setlength{\parindent}{\thm@parindent}\setlength{\parskip}{\thm@parskip}\itshape,
]{italic}

\declaretheoremstyle[% Upright style
    spaceabove=\thm@spacing,
    spacebelow=\thm@spacing,
    notefont=\bfseries,
    postheadspace=\thm@postheadspace,
    bodyfont=\setlength{\parindent}{\thm@parindent}\setlength{\parskip}{\thm@parskip}\upshape
]{upright}

\declaretheoremstyle[% Proof Style
    spaceabove=\thm@spacing,
    spacebelow=\thm@spacing,
    headfont=\mdseries\scshape,
    notefont=\mdseries\scshape,
    notebraces={\mbox{}\mbox{}},
    postheadspace=\thm@postheadspace,
    bodyfont=\setlength{\parindent}{\thm@parindent}\setlength{\parskip}{\thm@parskip}\upshape
]{myProof}

\providecommand{\qedsymbol}{}
\renewcommand{\qedsymbol}{\raisebox{0.08em}{\ensuremath{\mdwhtsquare}}} %Sets the QED symbol

\makeatother

% ======================================================== %
% ----------------- Theorem Declarations ----------------- %
% ======================================================== %

% ------ Theorem  ------ %
\declaretheorem[style=italic, parent=section, numberwithin=section, refname={Theorem, Theorems}]{theorem}
\declaretheorem[style=italic, numbered=no, title=Theorem]{theorem*}

% ------ Lemma ------ %
\declaretheorem[style=italic, sibling=theorem, numberlike=theorem, refname={Lemma, Lemmas}]{lemma}
\declaretheorem[style=italic, numbered=no, title=Lemma]{lemma*}

% ------ Corollary ------ %

\declaretheorem[style=italic, sibling=theorem, numberlike=theorem, refname={Corollary, Corollaries}]{corollary}
\declaretheorem[style=italic, numbered=no, title=Corollary]{corollary*}

% ------ Preposition ------ %
\declaretheorem[style=italic, sibling=theorem, numberlike=theorem, refname={Proposition, Propositions}]{proposition}
\declaretheorem[style=italic, numbered=no, title=Proposition]{proposition*}

% ------ Definition ------ %
\declaretheorem[style=upright, sibling=theorem, numberlike=theorem, refname={Definition, Definitions}]{definition}
\declaretheorem[style=upright, numbered=no, title=Definition]{definition*}

% ------ Remark ------ %
\declaretheorem[style=upright, sibling=theorem, numberlike=theorem, refname={Remark, Remarks}]{remark}
\declaretheorem[style=upright, numbered=no, title=Remark]{remark*}

% ------ Example ------ %
\declaretheorem[style=upright, sibling=theorem, numberlike=theorem, refname={Example, Examples}]{example}
\declaretheorem[style=upright, numbered=no, title=Example]{example*}

% ------ Claim ------ %
\declaretheorem[style=italic, numbered=no, title=Claim]{claim}

% ------ Proof ------ %
\declaretheorem[style=myProof, numbered=no,prefoothook={\qed}, title=Proof]{myProof}
\renewenvironment{proof}[1][]{\ifblank{#1}{\begin{myProof}}{\begin{myProof}[#1]}}{\end{myProof}}


% ------ Problem Solution ------ %
\xdeclaretheorem[style=italic, title=Problem, refname={Problem, Problems}]{0pt}{0pt}{problem}
\declaretheorem[style=myProof, numbered=no, prefoothook={\qed}, title=Solution]{solution}

% ------ Theorem Items ------ %
\newlist{thmitems}{enumerate}{1}
\setlist[thmitems]{label={\normalfont(\alph*)}, ref={\alph*}}
\labelcrefformat{thmitemsi}{{\upshape(#2{#1}#3)}}