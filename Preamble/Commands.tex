% ======================================================== %
% --------------------- Text Commands -------------------- %
% ======================================================== %

\newcommand{\ie}{i.e\xperiodcomma}
\newcommand{\eg}{e.g\xperiodcomma}
\newcommand{\etal}{et~al\xperiod}
\newcommand{\etc}{etc\xperiod}
\newcommand{\keyterm}[1]{\textbf{\emph{#1}}}

% ======================================================== %
% --------------------- Math Symbols --------------------- %
% ======================================================== %

\renewcommand{\emptyset}{\varnothing}

\newcommand{\dbar}{\textit{\mdseries đ}}
\newcommand{\updbar}{\textup{\mdseries đ}}

\newcommand{\defeq}{\mathrel{\vcentcolon \! =}}
\newcommand{\eqdef}{\mathrel{= \! \vcentcolon}}

\newcommand{\blankvar}[1][1ex]{{\mathmakebox[#1][c]{\cdot}}} % Placeholder Dot

\newcommand{\Naturals}[1][]{\ifblank{#1}{\mathbb{N}}{\mathbb{N}^{#1}}}
\newcommand{\Integers}[1][]{\ifblank{#1}{\mathbb{Z}}{\mathbb{Z}^{#1}}}
\newcommand{\Rationals}[1][]{\ifblank{#1}{\mathbb{Q}}{\mathbb{Q}^{#1}}}
\newcommand{\Reals}[1][]{\ifblank{#1}{\mathbb{R}}{\mathbb{R}^{#1}}}
\newcommand{\Complexs}[1][]{\ifblank{#1}{\mathbb{C}}{\mathbb{C}^{#1}}}

\newcommand{\mathp}{\mathclose{\,\text{.}}}% Text period in math
\newcommand{\mathc}{\mathclose{\,\text{,}}}% Text Comma in math

% ======================================================== %
% ------------------ Physical Constants ------------------ %
% ======================================================== %

\newcommand{\PlanckConst}{h} % Planck Constant
\newcommand{\rPlanckConst}{\hslash} % Reduced Planck Constant
\newcommand{\BoltzmanK}{k_B}

% ======================================================== %
% -------------------- Misc. Functions ------------------- %
% ======================================================== %

\newcommand{\func}[4][\to]{#2\mathpunct{:} #3 #1 #4} % Function Definition
\newcommand{\setcomp}[1]{{#1}^{c}} %Set Compliments
\newcommand{\vb}[1]{{\symbf{#1}}}

\providecommand\given{}
\newcommand\SetSymbol[1][]{\nonscript\:#1\vert\allowbreak\nonscript\:\mathopen{}}
\DeclarePairedDelimiterX{\Set}[1]\{\}{\renewcommand\given{\SetSymbol[\delimsize]}#1}

% ======================================================== %
% --------------- Misc. Paired Delimiters ---------------- %
% ======================================================== %

\DeclarePairedDelimiter{\INTERNALparen}{\lparen}{\rparen}
\DeclarePairedDelimiter{\INTERNALbrack}{\lbrack}{\rbrack}
\newcommand{\fparen}[1]{\INTERNALparen*{#1}}
\newcommand{\fbrack}[1]{\INTERNALbrack*{#1}}

\DeclarePairedDelimiterX\commutator[2]{[}{]}{\ifblank{#1}{\blankvar[1.2ex]}{#1}\ifblank{#2}{{,}\blankvar[1.2ex]}{,#2}} % Commutator
\DeclarePairedDelimiterX\innerp[2]{\langle}{\rangle}{\ifblank{#2}{\blankvar[1.2ex]}{#2}\ifblank{#1}{{,}\blankvar[1.2ex]}{,#1}} % Inner Product
\DeclarePairedDelimiterX\norm[1]{\lVert}{\rVert}{\ifblank{#1}{\blankvar}{#1}} % Vector Norm
\DeclarePairedDelimiterX\abs[1]{\lvert}{\rvert}{\ifblank{#1}{\blankvar}{#1}} % Absolute Value
\DeclarePairedDelimiterX\eval[1]{.}{\rvert}{#1} % Evaluated at
\reDeclarePairedDelimiterInnerWrapper\eval{nostarnonscaled}{#2#3}

% ======================================================== %
% ------------------- Dirac Notation --------------------- %
% ======================================================== %

\DeclarePairedDelimiter\ket{\lvert}{\rangle}
\DeclarePairedDelimiter\bra{\langle}{\rvert}
\DeclarePairedDelimiterX\braket[2]{\langle}{\rangle}{#1 \delimsize\vert\mathopen{}#2}
\DeclarePairedDelimiterX\ketbra[2]{\lvert}{\rvert}{#1\delimsize\rangle\delimsize\langle\mathopen{}#2}
\DeclarePairedDelimiterX\mel[3]{\langle}{\rangle}{#1\delimsize\vert\mathopen{}\,#2\,\delimsize\vert\mathopen{}#3}
\DeclarePairedDelimiterX\expval[1]{\langle}{\rangle}{#1}

% ======================================================== %
% ---------------------- Derivatives --------------------- %
% ======================================================== %

\DeclareDifferential{\xdif}{\updbar} %Inexact Differential
\DeclareDifferential{\pathdif}{\mathcal{D}} % Functional Measure
