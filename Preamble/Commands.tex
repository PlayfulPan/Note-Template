% ======================================================== %
% --------------------- Text Commands -------------------- %
% ======================================================== %

\newcommand{\ie}{i.e\xperiodcomma}
\newcommand{\cf}{cf\xperiod}
\newcommand{\eg}{e.g\xperiodcomma}
\newcommand{\etal}{et~al\xperiod}
\newcommand{\etc}{etc\xperiod}
\newcommand{\keyterm}[1]{\textbf{\emph{#1}}}

% ======================================================== %
% --------------------- Math Symbols --------------------- %
% ======================================================== %

\renewcommand{\emptyset}{\varnothing}

\newcommand{\dbar}{\textit{\mdseries đ}}
\newcommand{\updbar}{\textup{\mdseries đ}}

\newcommand{\defeq}{\mathrel{\vcentcolon \! =}}
\newcommand{\eqdef}{\mathrel{= \! \vcentcolon}}

\newcommand{\blankvar}[1][1ex]{{\mathmakebox[#1][c]{\cdot}}} % Placeholder Dot
\newcommand{\hodge}{\mathord{\vcenter{\hbox{\(\scriptstyle \star\)}}}} % Hodge Star

\newcommand{\Naturals}[1][]{\ifblank{#1}{\mathbb{N}}{\mathbb{N}^{#1}}}
\newcommand{\Integers}[1][]{\ifblank{#1}{\mathbb{Z}}{\mathbb{Z}^{#1}}}
\newcommand{\Rationals}[1][]{\ifblank{#1}{\mathbb{Q}}{\mathbb{Q}^{#1}}}
\newcommand{\Reals}[1][]{\ifblank{#1}{\mathbb{R}}{\mathbb{R}^{#1}}}
\newcommand{\Complexs}[1][]{\ifblank{#1}{\mathbb{C}}{\mathbb{C}^{#1}}}

\newcommand{\mathp}{\mathclose{\,\text{.}}}% Text period in math
\newcommand{\mathc}{\mathclose{\,\text{,}}}% Text Comma in math

% ======================================================== %
% ------------------ Physical Constants ------------------ %
% ======================================================== %

\newcommand{\PlanckConst}{h} % Planck Constant
\newcommand{\rPlanckConst}{\hslash} % Reduced Planck Constant
\newcommand{\BoltzmanK}{k_B}

% ======================================================== %
% -------------------- Misc. Functions ------------------- %
% ======================================================== %

\newcommand{\func}[4][\to]{#2\mathpunct{:} #3 #1 #4} % Function Definition

\NewDocumentCommand\conj{sm}{% Complex Conjugate
    \IfBooleanTF#1
    {\overline{#2}}
    {\overbar{#2}}
}

\NewDocumentCommand\transpose{m}{#1^{\symsf{T}}} %Transpose
\NewDocumentCommand\chargeconj{m}{#1^{\symsf{c}}} %Charge Conjugate
\NewDocumentCommand\conjtrans{m}{#1^{\dagger}} % Conjugate Tranpose
\NewDocumentCommand\inverse{m}{#1^{-1}} % Inverse

\NewDocumentCommand\Cl{s}{% Clifford Algebra
    \IfBooleanTF#1
    {\mathbb{C}\mathrm{l}}
    {\mathrm{Cl}}
}



\providecommand\given{}
\newcommand\SetSymbol[1][]{\nonscript\:#1\vert\allowbreak\nonscript\:\mathopen{}}
\DeclarePairedDelimiterX{\Set}[1]\{\}{\renewcommand\given{\SetSymbol[\delimsize]}#1}

% ======================================================== %
% --------------- Misc. Paired Delimiters ---------------- %
% ======================================================== %

\DeclarePairedDelimiter{\INTERNALparen}{\lparen}{\rparen}
\DeclarePairedDelimiter{\INTERNALbrack}{\lbrack}{\rbrack}
\newcommand{\fparen}[1]{\INTERNALparen*{#1}}
\newcommand{\fbrack}[1]{\INTERNALbrack*{#1}}

\DeclarePairedDelimiterX\commutator[2]{[}{]}{\ifblank{#1}{\blankvar[1.2ex]}{#1}\ifblank{#2}{{,}\blankvar[1.2ex]}{,#2}} % Commutator
\DeclarePairedDelimiterX\acommutator[2]{\{}{\}}{\ifblank{#1}{\blankvar[1.2ex]}{#1}\ifblank{#2}{{,}\blankvar[1.2ex]}{,#2}} % Anti-Commutator
\DeclarePairedDelimiterX\innerp[2]{\langle}{\rangle}{\ifblank{#1}{\blankvar[1.2ex]}{#1}\ifblank{#2}{{,}\blankvar[1.2ex]}{,#2}} % Inner Product
\DeclarePairedDelimiterX\norm[1]{\lVert}{\rVert}{\ifblank{#1}{\blankvar}{#1}} % Vector Norm
\DeclarePairedDelimiterX\abs[1]{\lvert}{\rvert}{\ifblank{#1}{\blankvar}{#1}} % Absolute Value
\DeclarePairedDelimiterX\eval[1]{.}{\rvert}{#1} % Evaluated at
\reDeclarePairedDelimiterInnerWrapper\eval{nostarnonscaled}{#2#3}

% ======================================================== %
% ------------------- Dirac Notation --------------------- %
% ======================================================== %

\DeclarePairedDelimiter\ket{\lvert}{\rangle}
\DeclarePairedDelimiter\bra{\langle}{\rvert}
\DeclarePairedDelimiterX\braket[2]{\langle}{\rangle}{#1 \delimsize\vert\mathopen{}#2}
\DeclarePairedDelimiterX\ketbra[2]{\lvert}{\rvert}{#1\delimsize\rangle\delimsize\langle\mathopen{}#2}
\DeclarePairedDelimiterX\mel[3]{\langle}{\rangle}{#1\delimsize\vert\mathopen{}#2\delimsize\vert\mathopen{}#3}
\DeclarePairedDelimiterX\expval[1]{\langle}{\rangle}{#1}

% ======================================================== %
% ---------------------- Derivatives --------------------- %
% ======================================================== %

\DeclareDifferential{\xdif}{\updbar} %Inexact Differential
\DeclareDifferential{\pathdif}{\mathcal{D}} % Functional Measure
\DeclareDerivative{\fdv}{\delta} % Functional Derivative
\derivset{\pdv}[delims-eval=.|]

\NewDocumentCommand\lieder{mm}{\mathcal{L}_{#1}#2}

\NewDocumentCommand\exder{som}{% Exterior Derivative, starred gives codifferential, optional argument gives connection subscript
    \IfBooleanTF{#1}
    {
        \IfNoValueTF{#2}
        {
            \mathrm{d}\indices{^{\ast}}#3
        }
        {
            \mathrm{d}\indices*{^{\ast}_{\scriptscriptstyle (#2)}}#3
        }
    }
    {
        \IfNoValueTF{#2}
        {
            \mathrm{d}#3
        }
        {
            \mathrm{d}\indices{_{\scriptscriptstyle (#2)}}#3
        }
    }
}
\NewDocumentCommand\covder{som}{% Covariant Derivative, starred gives capital D, optional argument gives connection superscriptscript, mandatory argument is subscript
    \IfBooleanTF{#1}
    {
        \IfNoValueTF{#2}
        {
            D\indices{_{\!#3}}
        }
        {
            D\indices*{^{\scriptscriptstyle(#2)}_{\!#3}}
        }
    }
    {
        \IfNoValueTF{#2}
        {
            \nabla\indices{_{\!#3}}
        }
        {
            \nabla\indices*{^{\scriptscriptstyle(#2)}_{\!#3}}
        }
    }
}



% ======================================================== %
% ------------------- Dotted Letters --------------------- %
% ======================================================== %

\newcommand{\dalpha}{{\dot{\alpha}}}
\newcommand{\dbeta}{{\dot{\beta}}}
\newcommand{\dgamma}{{\dot{\gamma}}}
\newcommand{\ddelta}{{\dot{\delta}}}
\newcommand{\depsilon}{{\dot{\epsilon}}}
\newcommand{\dzeta}{{\dot{\zeta}}}
\newcommand{\deta}{{\dot{\eta}}}
\newcommand{\dtheta}{{\dot{\theta}}}
\newcommand{\diota}{{\dot{\iota}}}
\newcommand{\dkappa}{{\dot{\kappa}}}
\newcommand{\dlambda}{{\dot{\lambda}}}
\newcommand{\dmu}{{\dot{\mu}}}
\newcommand{\dnu}{{\dot{\nu}}}
\newcommand{\dxi}{{\dot{\xi}}}
\newcommand{\dpi}{{\dot{\pi}}}
\newcommand{\drho}{{\dot{\rho}}}
\newcommand{\dsigma}{{\dot{\sigma}}}
\newcommand{\dtau}{{\dot{\tau}}}
\newcommand{\dupsilon}{{\dot{\upsilon}}}
\newcommand{\dphi}{{\dot{\phi}}}
\newcommand{\dchi}{{\dot{\chi}}}
\newcommand{\dpsi}{{\dot{\psi}}}
\newcommand{\domega}{{\dot{\omega}}}

\newcommand{\dwedge}{\mathbin{\dot{\wedge}}}
